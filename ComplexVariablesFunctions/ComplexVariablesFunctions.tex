\documentclass[12pt, a4paper, twoside]{ctexbook}
\usepackage{amsmath, amsthm, amssymb, bm, graphicx, hyperref, mathrsfs, geometry, booktabs, makecell}
\hypersetup{colorlinks=true, linkcolor=black}
\geometry{left=2.0cm, top=2.0cm, bottom=2.0cm, right=2.0cm}

\title{{\Huge{\textbf{复变函数笔记}}}}
\author{南京工程学院\\Serein}
\date{\today}
\linespread{1.5}
\newtheorem{theorem}{定理}[section]
\newtheorem{definition}[theorem]{定义}
\newtheorem{lemma}[theorem]{引理}
\newtheorem{corollary}[theorem]{推论}
\newtheorem{example}[theorem]{例}
\newtheorem{proposition}[theorem]{命题}
\everymath{\displaystyle}
\begin{document}

\maketitle

\pagenumbering{roman}
\setcounter{page}{1}

\newpage
\pagenumbering{Roman}
\setcounter{page}{1}
\tableofcontents
\newpage
\setcounter{page}{1}
\pagenumbering{arabic}

\chapter{复数与复变函数}
\newpage

\section{复数的概念及基本知识}
\subsubsection*{概念}
\textbf{复数}:$z=x+\mathrm{i}y$

\textbf{实部}:$x=\mathrm{Re}\left(z\right)$

\textbf{虚部}:$y=\mathrm{Im}\left(z\right)$

\textbf{注意}:两个复数相等当且仅当他们实部与虚部分别相等;两个复数只要不同时为实数就不能比较大小

\subsubsection*{代数运算}
\textbf{交换律}:$z_1+z_2=z_2+z_1$,$z_1z_2=z_2z_1$

\textbf{结合律}:$\left(z_1+z_2\right)+z_3=z_1+\left(z_2+z_3\right)=z_1+z_2+z_3$,$\left(z_1z_2\right)z_3=z_1\left(z_2z_3\right)=z_1z_2z_3$

\textbf{分配律}:$\left(z_1+z_2\right)z_3=z_1z_3+z_2z_3$

\subsubsection*{性质}
$$
\overline{\overline{z}}=z,\overline{z_1\pm z_2}=\overline{z_1}\pm\overline{z_2},\overline{z_1z_2}=\overline{z_1}\cdot \overline{z_2},\overline{\left(\frac{z_1}{z_2}\right)}=\frac{\overline{z_1}}{\overline{z_2}}
$$
$$
z\overline{z}=\left[\mathrm{Re}\left(z\right)\right]^2+\left[\mathrm{Im}\left(z\right)\right]^2
$$
$$
\mathrm{Re}\left(z\right)=\frac{z+\overline{z}}{2},\mathrm{Im}\left(z\right)=\frac{z-\overline{z}}{2\mathrm{i}}
$$

\subsubsection*{复平面}

\textbf{模}
$$
r=\left| z \right|=\sqrt{x^2+y^2}\geqslant 0
$$
$$
\left| x \right|\leqslant \left| z \right|,\left| y \right|\leqslant \left| z \right|,\left| z \right|\leqslant \left| x \right|+\left| y \right|
$$
$$
\left| z \right|=\left| \overline{z} \right|,z\overline{z}=x^2+y^2=\left| z \right|^2=\left| z^2 \right|
$$
$$
\left| z_1+z_2 \right|\leqslant \left| z_1 \right|+\left| z_2 \right|,\left| z_1-z_2 \right|\geqslant \left| \left| z_1 \right|-\left| z_2 \right| \right|
$$

\textbf{辐角}
$$
\tan \left(\mathrm{Arg}z\right)=\frac{y}{x},x\ne 0
$$
$$
\mathrm{Arg}z=\theta_0+2k\pi\left(k=0,\pm 1,\pm 2,\cdots\right)
$$
在$\left(-\pi,\pi\right]$内的辐角称为\textbf{辐角$\mathrm{Arg}z$的主值},记作$\mathrm{arg}\left(z\right)$,且其唯一,我们有
$$
\mathrm{Arg}z=\mathrm{arg}z+2k\pi\left(k=0,\pm 1,\pm 2,\cdots\right)
$$
$$
\text{arg}z=\left\{ \begin{array}{l}
	\arctan \frac{y}{x},x>0,\left(\text{\uppercase\expandafter{\romannumeral1},\uppercase\expandafter{\romannumeral4}}\right)\\
	\arctan \frac{y}{x}+\pi ,x<0,y\geqslant 0,\left(\text{\uppercase\expandafter{\romannumeral2}}\right)\\
	\arctan \frac{y}{x}-\pi ,x<0,y<0,\left(\text{\uppercase\expandafter{\romannumeral3}}\right)\\
	\frac{\pi}{2},x=0,y>0\\
	-\frac{\pi}{2},x=0,y<0\\
\end{array} \right. 
$$

\textbf{三角表示}
$$
z=r\left(\cos \theta + \mathrm{i} \sin \theta \right),\left| z \right|=r,\mathrm{arg}z=\theta
$$

\textbf{指数表示}
$$
z=r\mathrm{e}^{\mathrm{i}\theta}
$$

\textbf{性质}
$$
\left| z_1z_2 \right|=\left| z_1 \right|\left| z_2 \right|,\text{Arg}\left( z_1z_2 \right) =\text{Arg}z_1+\text{Arg}z_2
$$
$$
\left| \frac{z_1}{z_2} \right|=\frac{\left| z_1 \right|}{\left| z_2 \right|},\text{Arg}\left( \frac{z_1}{z_2} \right) =\text{Arg}z_1-\text{Arg}z_2
$$
$$
z_1z_2=r_1r_2\text{e}^{\text{i}\left( \theta _1+\theta _2 \right)},\frac{z_1}{z_2}=\frac{r_1}{r_2}\text{e}^{\text{i}\left( \theta _1-\theta _2 \right)}
$$
$$
z^mz^n=z^{m+n},\left( z^m \right) ^n=z^{mn},\left( z_1z_2 \right) ^n=z_{1}^{n}z_{2}^{n}
$$
$$
\text{i}^1=\text{i,i}^2=-1,\text{i}^3=-\text{i,i}^4=1
$$
$$
z^n=r^n\left( \cos n\theta +\text{i}\sin n\theta \right) 
$$
$$
\left( \cos \theta +\text{i}\sin \theta \right) ^n=\cos n\theta +\text{i}\sin n\theta 
$$
$$
\left( r\text{e}^{\text{i}\theta} \right) ^n=r^n\text{e}^{\text{i}n\theta}
$$
$$
\sqrt[n]{z}=\sqrt[n]{r}\left( \cos \frac{\theta +2k\pi}{n}+\text{i}\sin \frac{\theta +2k\pi}{n} \right) 
$$
$$
\sqrt[n]{r\text{e}^{\text{i}\theta}}=\sqrt[n]{r}\text{e}^{\text{i}\frac{\theta +2k\pi}{n}}
$$
\begin{example}
	在复数范围内求解\\
	\hspace*{2em}(1)$z^2+2z+3=0$\\
	\hspace*{2em}(2)$z^3+8=0$\\
	\hspace*{1em}\textbf{解}\\
	\hspace*{2em}(1)
	$$
	z_{1,2}=\frac{-2\pm \sqrt{4-12}}{2}=\frac{-2\pm \sqrt{-8}}{2}=\frac{-2\pm 2\sqrt{2}\mathrm{i}}{2}=-1\pm \sqrt{2}\mathrm{i}
	$$
	\newpage
	(2)由于$-8=8\left(\cos \pi+\mathrm{i}\sin \pi\right)$
	$$
	z=\sqrt[3]{-8}=2\left(\cos \frac{\pi +2k\pi}{3}+\mathrm{i}\sin \frac{\pi+2k\pi}{3}\right),k=0,1,2
	$$
	$$
	z_1=2\left(\cos\frac{\pi}{3}+\mathrm{i}\sin\frac{\pi}{3}\right)=1+\sqrt{3}\mathrm{i}
	$$
	$$
	z_2=2\left(\cos\pi+\mathrm{i}\sin\pi\right)=-2
	$$
	$$
	z_3=2\left(\cos\frac{5\pi}{3}+\mathrm{i}\sin\frac{5\pi}{3}\right)=1-\sqrt{3}\mathrm{i}
	$$
\end{example}

\section{复变函数}
\begin{theorem}
	函数$f\left(z\right)=u\left(x,y\right)+\mathrm{i}v\left(x,y\right)$在点$z_0=x_0+\mathrm{i}y_0$连续的充要条件是$u=u\left(x,y\right)$和$v=v\left(x,y\right)$在点$\left(x_0,y_0\right)$连续
\end{theorem}
\begin{example}
	$$
	\lim_{z\to1}\frac{z\overline{z}+z-\overline{z}-1}{z^2-1}=\lim_{z\to 1}\frac{\left(z-1\right)\left(\overline{z}+1\right)}{\left(z+1\right)\left(z-1\right)}=\lim_{z\to 1}\frac{\overline{z}+1}{z+1}=1
	$$
\end{example}
\begin{example}
	$$
	\lim_{z\to -1}\frac{\left|z^2\right|+2\mathrm{Re}\left(z\right)+1}{z^2-1}=\lim_{z\to -1}\frac{z\overline{z}+z+\overline{z}+1}{z^2-1}=\lim_{z\to -1}\frac{\left(z+1\right)\left(\overline{z}+1\right)}{\left(z+1\right)\left(z-1\right)}=\lim_{z\to -1}\frac{\overline{z}+1}{z-1}=0
	$$
\end{example}
\begin{example}
	函数$\omega=3z$和$\omega=z^3$分别将$z$平面上的区域
	$$
	\left\{ \begin{array}{l}
		\left| z \right|<1\\
		\left| \mathrm{arg}z \right|<\frac{\pi}{6}\\
	\end{array} \right. 
	$$
	\hspace*{2em}映射成$\omega$平面上的何种区域\\
	\hspace*{1em}\textbf{解}
	\hspace*{2em}
	$$
	\omega=3z\Rightarrow\left\{ \begin{array}{l}
		\left| \omega \right|<3\\
		\left| \mathrm{arg}z \right|<\frac{\pi}{6}\\
	\end{array} \right. 
	$$
	$$
	\omega=z^3 \Rightarrow\left\{ \begin{array}{l}
		\left| \omega \right|<1^3=1\\
		\left| \mathrm{arg}z \right|<\frac{\pi}{6}\times 3=\frac{\pi}{2}\\
	\end{array} \right. 
	$$
\end{example}

\chapter{解析函数}
\newpage
\section{解析函数的概念}
\begin{theorem}
	如果函数$f\left(z\right)$在$z_0$及$z_0$的某个邻域内可导,则\textbf{$f\left(z\right)$在$z_0$解析};如果$f\left(z\right)$在区域$D$内每一点都解析,则\textbf{$f\left(z\right)$在$D$内解析},$f\left(z\right)$为$D$内的\textbf{解析函数}
\end{theorem}
\section{函数解析的充要条件}
\begin{theorem}
	设函数$f\left(z\right)=u\left(x,y\right)+\mathrm{i}v\left(x,y\right)$定义在$D$内,$w=f\left(z\right)=u+\mathrm{i}v$在$D$内一点$z=x+\mathrm{i}y$可导的充要条件是$u\left(x,y\right)$与$v\left(x,y\right)$在点$\left(x,y\right)$可微,并且在该点满足\textbf{柯西-黎曼方程(C-R方程)}
	$$
	\frac{\partial u}{\partial x}=\frac{\partial v}{\partial y},\frac{\partial u}{\partial y}=-\frac{\partial v}{\partial x}
	$$
\end{theorem}
\begin{theorem}
	函数$f\left(z\right)=u\left(x,y\right)+\mathrm{i}v\left(x,y\right)$在区域$D$内解析的充要条件是$u\left(x,y\right)$与$v\left(x,y\right)$在$D$内可微,并且满足C-R方程
\end{theorem}
\textbf{提示:}求函数$f\left(z\right)$的导数,有两种情况:\\
(1)若表达式为关于$z$的式子,则直接求导即可\\
(2)若表达式为关于$x$与$y$的式子,则
$$
f'\left(z\right)=\frac{\partial u}{\partial x}+\mathrm{i}\frac{\partial v}{\partial x}=\frac{1}{\mathrm{i}}\frac{\partial u}{\partial y}+\frac{\partial v}{\partial y}
$$

\textbf{强调:}$f\left(z\right)=\overline{z}$在复平面内处处不可导,处处不解析
\section{初等函数}
\textbf{指数函数}
$$
\exp\left(z\right)=\mathrm{e}^z=\mathrm{e}^x\left(\cos y+\mathrm{i}\sin y\right),T=2k\pi\mathrm{i}
$$
$$
\mathrm{Arg}\left(\mathrm{e}^z\right)=y+2k\pi
$$
$$
\mathrm{e}^{z_1}\cdot\mathrm{e}^{z_2}=\mathrm{e}^{z_1+z_2},\left(\mathrm{e}^z\right)'=\mathrm{e}^z
$$

\textbf{对数函数}
$$
w=u+\mathrm{i}v=\mathrm{Ln}z=\ln \left| z \right| +\mathrm{iArg}z
$$
$\mathrm{Ln}z$的主值为
$$
\ln z = \ln \left|z\right|+\mathrm{iarg}z
$$
$$
\mathrm{Ln}z=\ln z +2k\pi\mathrm{i}
$$
需要注意以下几点
$$
\mathrm{Ln}\left(z_1z_2\right)=\mathrm{Ln}z_1+\mathrm{Ln}z_2,\mathrm{Ln}\left(\frac{z_1}{z_2}\right)=\mathrm{Ln}z_1-\mathrm{Ln}z_2
$$
$$
\mathrm{Ln}z+\mathrm{Ln}z\neq 2\mathrm{Ln}z,\mathrm{Ln}z^n\neq n\mathrm{Ln}z,\mathrm{Ln}\sqrt[n]{z}\neq \frac{1}{n}\mathrm{Ln}z
$$
$$
\left(\ln z\right)'=\frac{1}{z},\left(\mathrm{Ln}z\right)'=\frac{1}{z}
$$
$\ln z$以及$\mathrm{Ln}z$各分支在除原点和负实轴的复平面内解析

\textbf{幂函数}
$$
w=z^a=\mathrm{e}^{a\mathrm{Ln}z}=\mathrm{e}^{a\ln z}\mathrm{e}^{2k\pi a \mathrm{i}}
$$
当$a$为整数时
$$
w=\mathrm{e}^{a\ln z}
$$

\textbf{三角函数}
$$
\sin z=\frac{\mathrm{e}^{\mathrm{i}z}-\mathrm{e}^{-\mathrm{i}z}}{2\mathrm{i}}
$$
$$
\cos z=\frac{\mathrm{e}^{\mathrm{i}z}+\mathrm{e}^{-\mathrm{i}z}}{2}
$$
$$
\mathrm{e}^{\mathrm{i}z}=\cos z+\mathrm{i}\sin z
$$
(1)$\sin z$与$\cos z$,$T=2\pi$\\
(2)$\sin z$为奇函数,$\cos z$为偶函数\\
(3)$\sin z$与$\cos z$,对应的实三角函数恒等式、三角函数诱导公式等仍然成立\\
(4)$\sin z=0\Rightarrow z=\frac{\pi}{2}+k\pi$,$\cos z=0\Rightarrow z=k\pi$,$k=0,\pm 1,\pm 2,\cdots$\\
(5)$\sin z$与$\cos z$在复平面内处处可微,且$\left(\sin z\right)'=\cos z$,$\left(\cos z\right)'=-\sin z$\\
(6)$\left|\sin z\right|$与$\left|\cos z\right|$无界
\begin{example}
	指出函数
	$$
	\omega = \frac{1}{z^2+2z+5}
	$$
	\hspace*{2em}的解析性区域,并在该区域内求出其导数\\
	\hspace*{1em}\textbf{解}\\
	\hspace*{2em}当$z^2+2z+5=0$时
	$$
	z_{1,2}=\frac{-2\pm \sqrt{20-4}\mathrm{i}}{2}=-1\pm2\mathrm{i}
	$$
	\hspace*{2em}因此
	$$
	\omega = \frac{1}{z^2+2z+5}
	$$
	\hspace*{2em}解析性区域为除$z=-1\pm2\mathrm{i}$的复平面
	$$
	\omega ' = -\frac{1}{\left(z^2+2z+5\right)^2}\left(2z+2\right)= -\frac{2z+2}{\left(z^2+2z+5\right)^2},z\neq -1\pm2\mathrm{i}
	$$
\end{example}
\begin{example}
	求函数
	$$
	f\left(z\right)=x^3-y^3+2x^2y^2\mathrm{i}
	$$
	\hspace*{2em}与
	$$
	f\left(z\right)=\left|z^2\right|z
	$$
	\hspace*{2em}在何处可导,以及其解析性区域\\
	\hspace*{1em}\textbf{解}\\
	\hspace*{2em}(1)
	$$
	f\left(z\right)=x^3-y^3+2x^2y^2\mathrm{i},u\left(x,y\right)=x^3-y^3,v\left(x,y\right)=2x^2y^2
	$$
	$$
	\frac{\partial u}{\partial x}=3x^2,\frac{\partial u}{\partial y}=-3y^2
	$$
	$$
	\frac{\partial v}{\partial x}=4xy^2,\frac{\partial v}{\partial y}=4x^2y
	$$
	\hspace*{2em}令
	$$
	\left\{ \begin{array}{l}
		3x^2=4x^2y\\
		-3y^2=-4xy^2\\
	\end{array} \right. 
	$$
	\hspace*{2em}得
	$$
	\left\{ \begin{array}{l}
		x_1=0\\
		y_1=0\\
	\end{array} \right. 
	or
	\left\{ \begin{array}{l}
		x_2=\frac{3}{4}\\
		y_2=\frac{3}{4}\\
	\end{array} \right. 
	$$
	\hspace*{2em}因此
	$$
	f\left(z\right)=x^3-y^3+2x^2y^2\mathrm{i}
	$$
	\hspace*{2em}在$z=0$与$z=\frac{3}{4}+\frac{3}{4}\mathrm{i}$处可导,在复平面内处处不解析\\
	\hspace*{2em}(2)
	令$z=x+y\mathrm{i}$,则$\left|z^2\right|=\sqrt{\left(x^2-y^2\right)^2+4x^2y^2}=x^2+y^2$
	因此
	$$
	f\left(z\right)=\left(x^2+y^2\right)\left(x+\mathrm{i}y\right)=x^3+x^2y\mathrm{i}+xy^2+y^3\mathrm{i}=x^3+xy^2+\left(x^2y+y^3\right)\mathrm{i}
	$$
	$$
	u\left(x,y\right)=x^3+xy^2,v\left(x,y\right)=x^2y+y^3
	$$
	\hspace*{2em}因此
	$$
	\frac{\partial u}{\partial x}=3x^2+y^2,\frac{\partial u}{\partial y}=2xy
	$$
	$$
	\frac{\partial v}{\partial x}=2xy,\frac{\partial v}{\partial y}=x^2+3y^2
	$$
	\hspace*{2em}要使
	$$
	\frac{\partial u}{\partial x}=\frac{\partial v}{\partial y},\frac{\partial u}{\partial y}=-\frac{\partial v}{\partial x}
	$$
	\hspace*{2em}成立,则当且仅当$x=y=0$时成立\\
	\hspace*{2em}因此$f\left(z\right)$在$z=0$处可导,在复平面内处处不解析
\end{example}

\chapter{复变函数的积分}
\newpage
\section{复变函数积分的性质及计算方法}
计算复变函数积分类似于实数域内第二类曲线积分,当被积函数内出现积分路径,可以将积分路径带入计算
\begin{example}
	设$C$是从$2\to0$的上半圆周:$\left|z-1\right|=1$,求
	$$
	\int_C\left(1+\left|z-1\right|\right)\mathrm{d}z
	$$
	\hspace*{1em}\textbf{解}
    \begin{eqnarray}
        \int_C\left(1+\left|z-1\right|\right)\mathrm{d}z&=&\int_C\mathrm{d}z+\int_C\left|z-1\right|\mathrm{d}z \nonumber      \\
        ~&=&\int_C\mathrm{d}z+\int_C\mathrm{d}z  \nonumber    \\
        ~&=&2\int_C\mathrm{d}z \nonumber		\\
		~&=&2\int_2^0\mathrm{d}\left(x+\mathrm{i}y\right) \nonumber  \\
		~&=&-4 \nonumber
    \end{eqnarray}
\end{example}

\begin{example}
	计算$\int_C z\mathrm{d}z$,其中$C$为从原点到$3+4\mathrm{i}$的直线段\\
	\hspace*{1em}\textbf{解}\\
    \hspace*{2em}得参数方程
	$$
	\left\{ \begin{array}{l}
		x=x\\
		y=\frac{4}{3}x\\
	\end{array},x:0\rightarrow 3 \right. 
	$$
	$$
	z=x+\mathrm{i}y=\left( x+\frac{4}{3}x\right) \mathrm{i}
	$$
	\hspace*{2em}则
    \begin{eqnarray}
        \int_C z\mathrm{d}z&=&\int_0^3\left(x+\frac{4}{3}x\mathrm{i}\right)\mathrm{d}\left(x+\frac{4}{3}x\mathrm{i}\right) \nonumber      \\
        ~&=&\int_0^3\left(1+\frac{4}{3}\mathrm{i}\right)^2x\mathrm{d}x \nonumber    \\
        ~&=&\frac{9}{2}\times \left(1+\frac{4}{3}\mathrm{i}\right)^2 \nonumber		\\
		~&=&-\frac{7}{2}+12\mathrm{i} \nonumber
    \end{eqnarray}
\end{example}
\textbf{重点}\\
对包含$z_0$的任意一条正向简单闭曲线$C$,有
$$
\oint_C{\frac{\text{d}z}{\left( z-z_0 \right) ^{n+1}}}=\left\{ \begin{array}{l}
	2\pi \text{i,}n=0\\
	0,n\neq 0\\
\end{array} \right. 
$$
\section{复变函数积分的基本定理}
\textbf{柯西-古萨基本定理}\\
如果函数$f\left(z\right)$在单连通区域$D$内处处解析,则函数沿$D$内的任意一条简单闭曲线$C$的积分值为零,即
$$
\oint_C f\left(z\right)\mathrm{d}z=0
$$

\textbf{原函数与不定积分}\\
与实数域计算方法类似,类似于牛顿-莱布尼茨公式
\begin{example}
	$$
	\int_{-\pi\mathrm{i}}^{3\pi\mathrm{i}}{\mathrm{e}^{2z}}\mathrm{d}z=\left. \frac{1}{2}\mathrm{e}^{2z} \right|_{-\pi \mathrm{i}}^{3\pi \mathrm{i}}=\frac{1}{2}\left( \mathrm{e}^{6\pi\mathrm{i}}-\mathrm{e}^{-2\pi\mathrm{i}}\right)=\frac{1-1}{2}=0
	$$
\end{example}

\textbf{复合闭路定理}
$$
\oint_C{f\left( z \right) \text{d}z}=\oint_{C_1}{f\left( z \right) \text{d}z}
$$
$$
\oint_C{f\left( z \right) \text{d}z}=\sum_{k=1}^n{\oint_{C_k}{f\left( z \right) \text{d}z}}
$$

\begin{example}
	计算
	$$
	\oint_{\Gamma}{\frac{1}{z^2-z}\mathrm{d}z}
	$$
	\hspace*{2em}其中$\Gamma$为包含圆周$\left|z\right|=1$在内的任意一条正向简单闭曲线\\
	\hspace*{1em}\textbf{解}\\
	\hspace*{2em}以$0$与$1$为中心,在$\Gamma$内作正向圆周$C_1$与$C_2$,且两者互不相交,互不包含,则
	\begin{eqnarray}
        \oint_{\Gamma}{\frac{1}{z^2-z}\mathrm{d}z}&=&\oint_{C_1}{\frac{1}{z^2-z}\mathrm{d}z}+\oint_{C_2}{\frac{1}{z^2-z}\mathrm{d}z} \nonumber      \\
        ~&=&\oint_{C_1}{\left(\frac{1}{z-1}-\frac{1}{z}\right)\mathrm{d}z}+\oint_{C_2}{\left(\frac{1}{z-1}-\frac{1}{z}\right)\mathrm{d}z} \nonumber    \\
        ~&=&0-2\pi\mathrm{i}+2\pi\mathrm{i}-0 \nonumber	  \\
		~&=&0 \nonumber
    \end{eqnarray}
\end{example}


\textbf{复变函数积分的基本公式}
\begin{theorem}
	如果函数$f\left(z\right)$在区域$D$内处处解析,$C$为$D$内的任何一条正向简单闭曲线,它的内部完全含于$D$,$z_0$为$C$内的任意一点,则
	$$
	\oint_C{\frac{f\left( z \right)}{z-z_0}\mathrm{d}z=}2\pi \mathrm{i}f\left( z_0 \right) 
	$$
\end{theorem}
\newpage
\begin{example}
	计算
	$$
	\frac{1}{2\pi\mathrm{i}}\oint_{\left|z\right|=4}{\frac{\sin 2z}{z}\mathrm{d}z}
	$$
	\hspace*{2em}沿圆周正向\\
	\hspace*{1em}\textbf{解}\\
	\hspace*{2em}由于$\sin 2z$在$\left|z\right|=4$内解析,则
	$$
	\frac{1}{2\pi\mathrm{i}}\oint_{\left|z\right|=4}{\frac{\sin 2z}{z}\mathrm{d}z}=\left.\sin 2z \right|_{z=0}=0
	$$
\end{example}
\begin{example}
	计算
	$$
	\oint_{\left|z+2\right|=1}{\frac{z}{z^2-4}}\mathrm{d}z
	$$
	\hspace*{2em}沿圆周正向\\
	\hspace*{1em}\textbf{方法一:利用柯西积分公式}\\
	\begin{eqnarray}
        \oint_{\left|z+2\right|=1}{\frac{z}{z^2-4}}\mathrm{d}z&=&\oint_{\left| z+2 \right|=1}{\frac{z}{\left( z-2 \right) \left( z+2 \right)}\mathrm{d}z} \nonumber      \\
        ~&=&\oint_{\left| z+2 \right|=1}{\frac{\frac{z}{z-2}}{z+2}\mathrm{d}z} \nonumber    \\
        ~&=&2\pi\mathrm{i}\cdot\left.\frac{z}{z-2}\right|_{z=-2} \nonumber	  \\
		~&=&2\pi\mathrm{i}\cdot \frac{-2}{-4} \nonumber  \\
		~&=&\pi\mathrm{i} \nonumber
	\end{eqnarray}
	\hspace*{1em}\textbf{方法二:利用复合闭路定理}\\
	\begin{eqnarray}
        \oint_{\left|z+2\right|=1}{\frac{z}{z^2-4}}\mathrm{d}z&=&\oint_{\left| z+2 \right|=1}{\frac{z}{\left( z-2 \right) \left( z+2 \right)}\mathrm{d}z} \nonumber      \\
        ~&=&\oint_{\left| z+2 \right|=1}{\frac{1}{2}\left( \frac{1}{z-2}+\frac{1}{z+2} \right) \mathrm{d}z} \nonumber    \\
        ~&=&\frac{1}{2}\oint_{\left| z+2 \right|=1}{\frac{1}{z-2}\mathrm{d}z}+\frac{1}{2}\oint_{\left| z+2 \right|=1}{\frac{1}{z+2}\mathrm{d}z} \nonumber	  \\
		~&=&0+\frac{1}{2}\times2\pi\mathrm{i} \nonumber  \\
		~&=&\pi\mathrm{i} \nonumber
	\end{eqnarray}
\end{example}
\textbf{解析函数的高阶导数公式}
\begin{theorem}
	$$
	\oint_C{\frac{f\left( z \right)}{\left( z-z_0 \right) ^{n}}\mathrm{d}z}=2\pi \mathrm{i}\cdot \frac{f^{\left( n-1 \right)}\left( z_0 \right)}{\left(n-1\right)!}
	$$
\end{theorem}
\begin{example}
	$$
	\oint_{\left|z\right|=3}{\frac{3z^2+2z-1}{\left(z-1\right)^3}\mathrm{d}z}=2\pi\mathrm{i}\cdot \frac{\left.\left(3z^2+2z-1\right)''\right|_{z=1}}{\left(3-1\right)!}=6\pi\mathrm{i}
	$$
\end{example}
\section{解析函数与调和函数}
\textbf{调和函数}\\
如果二元实函数$\varphi \left( x,y \right)$在区域$D$内有二阶连续偏导数,并且满足拉普拉斯方程
$$
\frac{\partial ^2 \varphi}{\partial x^2}+\frac{\partial ^2 \varphi}{\partial y^2}=0
$$
则称$\varphi \left( x,y \right)$为区域$D$的调和函数\\

\textbf{解析函数与调和函数的关系}
\begin{theorem}
	任何在区域$D$内解析的函数,它的实部与虚部均为区域$D$内的调和函数,且其虚部为实部的共轭调和函数;我们把使得$u+\mathrm{i}v$在区域$D$内构成解析函数的调和函数$v$称为$u$的共轭调和函数
\end{theorem}

\begin{example}
	证明$u\left(x,y\right)=2\left(x-1\right)y$为调和函数,并求其共轭调和函数$v\left(x,y\right)$和由他们构成的解析函数$f\left(z\right)=u+\mathrm{i}v$\\
	\hspace*{1em}\textbf{解}\\
	\hspace*{2em}(1)\textbf{证明:}
	$$
	\frac{\partial u}{\partial x}=2y,
	\frac{\partial u}{\partial y}=2x-2
	$$
	$$
	\frac{\partial ^2 v}{\partial x^2}=0,
	\frac{\partial ^2 v}{\partial y^2}=0
	$$
	\hspace*{2em}则
	$$
	\frac{\partial ^2 \varphi}{\partial x^2}+\frac{\partial ^2 \varphi}{\partial y^2}=0
	$$
	\hspace*{2em}因此$u\left(x,y\right)=2\left(x-1\right)y$为调和函数
	\newpage
	(2)由于满足C-R方程,因此有
	$$
	\frac{\partial u}{\partial x}=\frac{\partial v}{\partial y}=2y
	$$
	\hspace*{2em}则
	$$
	v=y^2+g\left(x\right)
	$$
	\hspace*{2em}又由于
	$$
	\frac{\partial v}{\partial x}=-\frac{\partial u}{\partial y}=2-2x
	$$
	\hspace*{2em}且
	$$
	\frac{\partial v}{\partial x}=\left(y^2+g\left(x\right)\right)'_x=g'\left(x\right)
	$$
	\hspace*{2em}因此有
	$$
	g\left(x\right)=-x^2+2x+C
	$$
	\hspace*{2em}因此
	$$
	f\left(z\right)=2\left(x-1\right)y+\mathrm{i}\left(y^2-x^2+2x+C\right)
	$$
	\hspace*{2em}同时也可带入
	$$
	x=\frac{z+\overline{z}}{2},y=\frac{z-\overline{z}}{2\mathrm{i}}
	$$
	\hspace*{2em}得到
	$$
	f\left(z\right)=\mathrm{i}\left(-z^2+2z+C\right)
	$$
\end{example}

\chapter{级数}
\newpage
\section{复数项级数与幂级数}
\begin{definition}
	若级数$\sum\limits_{n=1}^{\infty}{\left|\alpha_n\right|}$收敛,则复数项级数$\sum\limits_{n=1}^{\infty}{\alpha_n}$绝对收敛;若级数$\sum\limits_{n=1}^{\infty}{\alpha_n}$收敛,而$\sum\limits_{n=1}^{\infty}{\left|\alpha_n\right|}$发散,则复数项级数$\sum\limits_{n=1}^{\infty}{\alpha_n}$条件收敛\\
\end{definition}
\begin{theorem}
	对幂级数$\sum\limits_{n=0}^{\infty}{c_nz^n}$\\
	(1)若$\lim\limits_{n\to\infty}{\left|\frac{c_{n+1}}{c_n}\right|}=\lambda\neq0$,则收敛半径$R=\frac{1}{\lambda}$\\
	(2)若$\lim\limits_{n\to\infty}{\sqrt[n]{\left|c_n\right|}}=\lambda\neq0$,则收敛半径$R=\frac{1}{\lambda}$
\end{theorem}
\textbf{收敛半径}
$$
R=\lim\limits_{n\to\infty}{\left|\frac{c_n}{c_{n+1}}\right|}
$$
\section{泰勒级数}
\begin{theorem}
	设函数$f\left(z\right)$在区域$D$内解析,$z_0$为区域$D$内一点,$d$为$z_0$到$D$的边界上各点的最短距离,则当$\left|z-z_0\right|\leqslant d$时,$f\left(z\right)$可以展开成幂级数,且其唯一
	$$
	f\left(z\right)=\sum_{n=0}^{\infty}{\frac{f^{(n)}\left(z_0\right)}{n!}\left(z-z_0\right)^n}
	$$
\end{theorem}
\textbf{重要的泰勒展开式}
$$
\mathrm{e}^z=\sum_{n=0}^{\infty}\frac{z^n}{n!}=1+z+\frac{z^2}{2!}+\frac{z^3}{3!}+\cdots,R=+\infty
$$
$$
\sin z = \sum_{n=0}^{\infty}(-1)^n\frac{z^{2n+1}}{(2n+1)!}=z-\frac{z^3}{3!}+\frac{z^5}{5!}-\cdots,R=+\infty
$$
$$
\cos z = \sum_{n=0}^{\infty}(-1)^n\frac{z^{2n}}{(2n)!}=1-\frac{z^2}{2!}+\frac{z^4}{4!}-\cdots,R=+\infty
$$
$$
\ln\left(1+z\right)=\sum_{n=1}^{\infty}(-1)^{n-1}\frac{z^n}{n}=z-\frac{z^2}{2}+\frac{z^3}{3}-\cdots,\left|z\right|<1
$$
$$
\frac{1}{1+z}=\sum_{n=0}^{\infty}(-1)^nz^n=1-z+z^2-\cdots,\left|z\right|<1
$$
$$
\frac{1}{1-z}=\sum_{n=0}^{\infty}z^n=1+z+z^2+\cdots,\left|z\right|<1
$$

\textbf{收敛半径的求法}\\
对于收敛半径的求解,一般可利用上述求收敛半径的公式求解;若求函数在$z_0$处的泰勒展开式的收敛半径,则先找出该函数的奇点,求奇点与$z_0$的最小距离,该距离即为收敛半径
\begin{example}
	求
	$$
	f\left(z\right)=\frac{1}{1+z^2}
	$$
	\hspace*{2em}在$z=1$处的泰勒展开式的收敛半径\\
	\hspace*{1em}\textbf{解}\\
	\hspace*{2em}令$1+z^2=0$,得$z=\pm \mathrm{i}$,又$\left(0,\pm \mathrm{i}\right)$到$\left(1,0\right)$距离为$\sqrt{2}$,因此收敛半径$R=\sqrt{2}$
\end{example}
\begin{example}
	求级数
	$$
	\sum_{n=1}^{\infty}\left(-3\right)^nz^n
	$$
	\hspace*{2em}的收敛半径\\
	\hspace*{1em}\textbf{解}
	$$
	R=\lim_{n\to \infty}\left|\frac{\left(-3\right)^n}{\left(-3\right)^{n+1}}\right|=\frac{1}{3}
	$$
\end{example}
\section{敛散性的判断}
\textbf{等比级数}
$$
\sum_{n=1}^{\infty}a_1q^{n-1}
$$
\hspace*{3em}当$\left|q\right|<1$时,收敛\\
\hspace*{3em}当$\left|q\right|\geqslant1$时,发散\\

\textbf{调和级数}
$$
\sum_{n=1}^{\infty}\frac{1}{n}
$$
\hspace*{3em}发散\\

\textbf{$p$级数}
$$
\sum_{n=1}^{\infty}\frac{1}{n^p}
$$
\hspace*{3em}当$p\leqslant1$时,发散\\
\hspace*{3em}当$p>1$时,收敛\\

\begin{theorem}
	若
	$$
	\sum_{n=1}^{\infty}u_n
	$$
	收敛,则
	$$
	\lim_{n\to\infty}u_n=0
	$$
\end{theorem}
对于正项级数
$$
\sum_{n=1}^{\infty}u_n,\sum_{n=1}^{\infty}v_n
$$
有如下审敛法\\

\textbf{比较审敛法}
$$u_n\leqslant v_n$$
\hspace*{3em}(1)$\sum\limits_{n=1}^{\infty}v_n$收敛$\Rightarrow\sum\limits_{n=1}^{\infty}u_n$收敛\\
\hspace*{3em}(2)$\sum\limits_{n=1}^{\infty}u_n$发散$\Rightarrow\sum\limits_{n=1}^{\infty}v_n$发散\\
\hspace*{3em}简言之:大收则小收,小发则大发\\

\textbf{比较审敛法的极限形式}
$$
\lim_{n\to\infty}\frac{u_n}{v_n}=l
$$
\hspace*{3em}(1)若$0<l<+\infty$,则$\sum\limits_{n=1}^{\infty}u_n$,$\sum\limits_{n=1}^{\infty}v_n$同敛散\\
\hspace*{3em}(2)若$l=0$,则$\sum\limits_{n=1}^{\infty}v_n$收敛$\Rightarrow\sum\limits_{n=1}^{\infty}u_n$收敛\\
\hspace*{3em}(3)若$l=+\infty$,则$\sum\limits_{n=1}^{\infty}v_n$发散$\Rightarrow\sum\limits_{n=1}^{\infty}u_n$发散\\

\textbf{比值审敛法}
$$
\lim_{n\to \infty}\frac{u_{n+1}}{u_n}=\rho
$$
\hspace*{3em}(1)若$0<\rho<1$,则级数收敛\\
\hspace*{3em}(2)若$\rho>1$,则级数发散\\
\hspace*{3em}(3)若$\rho=1$,则级数敛散性不确定\\

\textbf{根值审敛法}
$$
\lim_{n\to\infty}\sqrt[n]{u_n}=\rho
$$
\hspace*{3em}(1)若$\rho<1$,则级数收敛\\
\hspace*{3em}(2)若$\rho>1$,则级数发散\\
\hspace*{3em}(3)若$\rho=1$,则级数敛散性不确定
\newpage
\textbf{交错级数敛散性的判别}
$$
\sum_{n=1}^{\infty}\left(-1\right)^{n-1}u_n
$$
若同时满足以下两个条件,则收敛
$$
\lim_{n\to \infty}u_n=0,u_n\geqslant u_{n+1}
$$

\textbf{拓展:}若用\textbf{比值}或\textbf{根值}审敛法判别,则当判定出$\sum\limits_{n=1}^{\infty}\left|u_n\right|$发散,则$\sum\limits_{n=1}^{\infty}u_n$发散

\section{洛朗级数}
\begin{example}
	将函数
	$$
	f\left(z\right)=\frac{1}{z-2}
	$$
	\hspace*{2em}在解析区域$1<\left|z-1\right|<+\infty$内展开成洛朗级数\\
	\hspace*{1em}\textbf{解}
	$$
	f\left(z\right)=\frac{1}{z-2}=\frac{1}{z-1-1}=\frac{1}{z-1}\cdot \frac{1}{1-\frac{1}{z-1}}
	$$
	\hspace*{2em}由于$1<\left|z-1\right|<+\infty$,则$\left|z-1\right|<1$,因此有
	$$
	f\left(z\right)=\frac{1}{z-1}\sum_{n=0}^{\infty}\left(\frac{1}{z-1}\right)^n=\sum_{n=1}^{\infty}\left(\frac{1}{z-1}\right)^n
	$$
\end{example}
\begin{example}
	将函数
	$$
	f\left(z\right)=\frac{1}{\left(z-1\right)\left(z-2\right)}
	$$
	\hspace*{2em}在解析区域\\
	\hspace*{2em}(1) $\left|z\right|<1$\\
	\hspace*{2em}(2) $1<\left|z\right|<2$\\
	\hspace*{2em}(3) $2<\left|z\right|<+\infty$\\
	\hspace*{2em}内展开成洛朗级数\\
	\hspace*{1em}\textbf{解}
	$$
	f\left(z\right)=\frac{1}{z-2}-\frac{1}{z-1}=\frac{1}{1-z}-\frac{1}{2-z}
	$$
	\hspace*{2em}(1)在$\left|z\right|<1$内
	$$
	f\left(z\right)=\frac{1}{1-z}-\frac{1}{2}\frac{1}{1-\frac{z}{2}}=\sum_{n=0}^{\infty}z^n-\frac{1}{2}\sum_{n=0}^{\infty}\left(\frac{z}{2}\right)^n
	$$
	\newpage
	(2)在$1<\left|z\right|<2$内,$\left|\frac{1}{z}\right|<1$,则
	$$
	\frac{1}{1-z}=-\frac{1}{z}\frac{1}{1-\frac{1}{z}}=-\frac{1}{z}\sum_{n=0}^{\infty}\left(\frac{1}{z}\right)^n
	$$
	\hspace*{2em}且$\left|\frac{z}{2}\right|<1$,则
	$$
	\frac{1}{2-z}=\frac{1}{2}\frac{1}{1-\frac{z}{2}}=\frac{1}{2}\sum_{n=0}^{\infty}\left(\frac{z}{2}\right)^n
	$$
	\hspace*{2em}因此
	$$
	f\left(z\right)=-\frac{1}{z}\sum_{n=0}^{\infty}\left(\frac{1}{z}\right)^n-\frac{1}{2}\sum_{n=0}^{\infty}\left(\frac{z}{2}\right)^n
	$$
	\hspace*{2em}(3)在$2<\left|z\right|<+\infty$内,$\left|\frac{1}{z}\right|<1$,$\left|\frac{2}{z}\right|<1$则
	$$
	\frac{1}{2-z}=-\frac{1}{z}\frac{1}{1-\frac{2}{z}}=-\frac{1}{z}\sum_{n=0}^{\infty}\left(\frac{2}{z}\right)^n
	$$
	\hspace*{2em}因此
	$$
	f\left(z\right)=-\frac{1}{z}\sum_{n=0}^{\infty}\left(\frac{1}{z}\right)^n+\frac{1}{z}\sum_{n=0}^{\infty}\left(\frac{2}{z}\right)^n
	$$
\end{example}
\begin{example}
	将函数
	$$
	f\left(z\right)=\frac{1}{z\left(1-z\right)^2}
	$$
	\hspace*{2em}分别在圆环域\\
	\hspace*{2em}(1) $0<\left|z\right|<1$ \\
	\hspace*{2em}(2) $0<\left|z-1\right|<1$ \\
	\hspace*{2em}内展开成洛朗级数\\
	\hspace*{1em}\textbf{解}\\
	\hspace*{2em}(1)在$0<\left|z\right|<1$内,有
	$$
	\frac{1}{1-z}=\sum_{n=0}^{\infty}z^n
	$$
	\hspace*{2em}则
	$$
	\frac{1}{\left(1-z\right)^2}=\left(\frac{1}{1-z}\right)'=\left(\sum_{n=0}^{\infty}z^n\right)'=\sum_{n=1}^{\infty}nz^{n-1}
	$$
	\hspace*{2em}因此
	$$
	f\left(z\right)=\frac{1}{z}\sum_{n=1}^{\infty}nz^{n-1}=\sum_{n=1}^{\infty}nz^{n-2}
	$$
	\newpage
	(2)在$0<\left|z-1\right|<1$内,
	$$
	\frac{1}{z}=\frac{1}{1+z-1}=\sum_{n=0}^{\infty}\left(-1\right)^n\left(z-1\right)^n
	$$
	\hspace*{2em}因此
	$$
	f\left(z\right)=\frac{1}{z}\cdot\frac{1}{\left(1-z\right)^2}=\sum_{n=0}^{\infty}\left(-1\right)^n\left(z-1\right)^{n-2}
	$$
\end{example}
\textbf{结论}\\
根据洛朗级数相关性质,有
$$
\oint_C{f\left(z\right)\mathrm{d}z}=2\pi\mathrm{i}c_{-1}
$$
\begin{example}
	求
	$$
	\oint_{\left|z\right|=\frac{1}{2}}{\frac{1}{z\left(1-z\right)^2}\mathrm{d}z}
	$$
	\hspace*{1em}\textbf{解}\\
	\hspace*{2em}函数
	$$
	f\left(z\right)=\frac{1}{z\left(1-z\right)^2}
	$$
	\hspace*{2em}在$0<\left|z\right|<1$内处处解析,且$\left|z\right|=\frac{1}{2}$在此圆环内,又由于
	$$
	f\left(z\right)=\frac{1}{z\left(1-z\right)^2}=\frac{1}{z}\left(\frac{1}{1-z}\right)'=\sum_{n=1}^{\infty}nz^{n-2}
	$$
	\hspace*{2em}因此
	$$
	\oint_{\left|z\right|=\frac{1}{2}}{\frac{1}{z\left(1-z\right)^2}\mathrm{d}z}=2\pi\mathrm{i}c_{-1}=2\pi\mathrm{i}
	$$
\end{example}

\chapter{留数}
\newpage
\section{孤立奇点}
如果函数$f\left(z\right)$在$z_0$处不解析,则$z_0$为$f\left(z\right)$的奇点
\begin{definition}
	如果函数$f\left(z\right)$在$z_0$处不解析,但在$z_0$的某个去心邻域$0<\left|z-z_0\right|<\delta$内处处解析,则$z_0$为$f\left(z\right)$的孤立奇点
\end{definition}
\begin{definition}
	设$z_0$为$f\left(z\right)$的孤立奇点,在$z_0$的某个去心邻域$0<\left|z-z_0\right|<\delta$内,函数$f\left(z\right)$展开成洛朗级数
	$$
	f\left(z\right)=\sum_{n=-\infty}^{\infty}c_{n}\left(z-z_0\right)^n
	$$
	(1)展开式中不含$\left(z-z_0\right)$的负幂次项$\Rightarrow z_0$为可去奇点\\
	(2)展开式中只有有限个$\left(z-z_0\right)$的负幂次项$\Rightarrow z_0$为极点,且若最高负幂次项次数为$-m$,则其为$m$级极点\\
	(3)展开式中有无穷个$\left(z-z_0\right)$的负幂次项$\Rightarrow z_0$为本性奇点
\end{definition}
\textbf{引申}\\
(1)$\lim\limits_{z\to z_0}{f\left(z\right)}$存在$\Rightarrow z_0$为可去奇点\\
(2)$\lim\limits_{z\to z_0}{f\left(z\right)}=\infty \Rightarrow z_0$为极点\\
(3)$\lim\limits_{z\to z_0}{f\left(z\right)}$不存在且不为$\infty \Rightarrow z_0$为本性奇点
\begin{example}
	求函数
	$$
	f\left(z\right)=\frac{z^3}{\left(z^2+1\right)\left(z+1\right)^2}
	$$
	\hspace*{2em}的极点,并指出是几级极点\\
	\hspace*{1em}\textbf{解}\\
	\hspace*{2em}$z=\pm\mathrm{i}$为一级极点,$z=-1$为二级极点
\end{example}
\textbf{说明}\\
若解析函数$f\left(z\right)$在$z_0$的某邻域内满足
$$
f\left(z_0\right)=f'\left(z_0\right)=\cdots =f^{(m-1)}\left(z_0\right)=0,f^{(m)}\left(z_0\right)\neq0
$$
则$z_0$为$f\left(z\right)$的$m$级零点
\newpage
\begin{example}
	求下列函数的极点,并指出是几级极点\\
	\hspace*{2em}(1) $f\left(z\right)=\frac{\mathrm{e}^z-1}{z^2}$\\
	\hspace*{2em}(2) $f\left(z\right)=\frac{1}{\sin z}$\\
	\hspace*{1em}\textbf{解}\\
	\hspace*{2em}(1)
	$$
	f\left(z\right)=\frac{\mathrm{e}^z-1}{z^2}=\frac{1}{z^2}\left(\sum_{n=0}^{\infty}\frac{z^n}{n!}-1\right)=\frac{1}{z^2}\left(z+\frac{z^2}{2!}+\frac{z^3}{3!}+\cdots\right)=\frac{1}{z}+\frac{1}{2!}+\frac{z}{3!}+\cdots
	$$
	\hspace*{2em}因此$z=0$为$f\left(z\right)$的一级极点\\
	\hspace*{1em}\textbf{另解}\\
	\hspace*{2em}由于对于分母$z=0$为$z^2$的二级零点,对于分子$z=0$为$\mathrm{e}^z-1$的一级零点,这是由于
	$$
	\left.\left(\mathrm{e}^z-1\right)\right|_{z=0}=0,\left.\left(\mathrm{e}^z-1\right)'\right|_{z=0}=1
	$$
	\hspace*{2em}因此$z=0$为$f\left(z\right)$的一级极点($2-1=1$)\\
	\hspace*{2em}(2)由于
	$$
	\sin\left(k\pi\right)=0,\left.\left[\sin\left(z\right)\right]'\right|_{k\pi}=\cos \left(k\pi\right)=\left(-1\right)^k\neq 0
	$$
	\hspace*{2em}因此$z=k\pi$为$\sin z$的一级零点,即$z=k\pi$为$f\left(z\right)=\frac{1}{\sin z}$的一级极点
\end{example}
\textbf{结论}\\
	\hspace*{2em}对于某函数$f\left(z\right)$,在$z_0$处,其为分子的$a$级零点,分母的$b$级零点,且$a<b$,则$z_0$为$f\left(z\right)$的$b-a$级极点\\
	\hspace*{2em}在计算几级极点时,需要将分子分母都因式分解,约去公因式,计算$z_0$为分子、分母的几级零点\\
	\hspace*{2em}计算某函数某一点为其几级零点时,首先因式分解到最简为几个整式相乘的形式,计算$z_0$为各部分的几级零点,最后相加得$z_0$为该函数的几级零点
\begin{example}
	求函数
	$$
	f\left(z\right)=\frac{z-\sin z}{z^6}
	$$
	\hspace*{2em}的极点,并指出是几级极点\\
	\hspace*{1em}\textbf{解}\\
	\hspace*{2em}对于分子,令$p\left(z\right)=z-\sin z$,
	$$
	p\left(0\right)=0,p'\left(0\right)=0,p''\left(0\right)=0,p'''\left(0\right)=1
	$$
	\hspace*{2em}显然$z=0$为$p\left(z\right)$的三级零点,又$z=0$为$z^6$的六级极点\\
	\hspace*{2em}因此$z=0$为$f\left(z\right)$的三级极点\\
\end{example}

\section{留数概念与计算}
$f\left(z\right)$在$z_0$处的留数为
$$
\mathrm{Res}\left[f\left(z\right),z_0\right],\mathrm{Res}\left[f\left(z_0\right)\right]
$$
$$
\mathrm{Res}\left[f\left(z\right),z_0\right]=\frac{1}{2\pi\mathrm{i}}\oint_C{f\left(z\right)\mathrm{d}z}=c_{-1}
$$
(1)若$z_0$为可去奇点,则
$$
\mathrm{Res}\left[f\left(z\right),z_0\right]=c_{-1}=0
$$
(2)若$z_0$为本性奇点,则展开洛朗级数,取$c_{-1}$\\
(3)若$z_0$为极点,有以下规则:\\
\hspace*{2em}\textbf{规则1:}若$z_0$为一级极点,则
$$
\mathrm{Res}\left[f\left(z\right),z_0\right]=\lim_{z\to z_0}{\left(z-z_0\right)f\left(z\right)}
$$
\hspace*{2em}\textbf{规则2:}若$z_0$为$m$级极点,则
$$
\mathrm{Res}\left[f\left(z\right),z_0\right]=\frac{1}{\left(m-1\right)!}\lim_{z\to z_0}\frac{\mathrm{d}^{m-1}}{\mathrm{d}z^{m-1}}\left[\left(z-z_0\right)^m f\left(z\right)\right]
$$
\hspace*{2em}\textbf{规则3:}设
$$
f\left(z\right)=\frac{P\left(z\right)}{Q\left(z\right)}
$$
$P\left(z\right)$与$Q\left(z\right)$在$z_0$解析,如果$P\left(z_0\right)\neq 0$,$Q\left(z_0\right)=0$,$Q'\left(z_0\right)\neq 0$,则$z_0$为$f\left(z\right)$的一级极点,且
$$
\mathrm{Res}\left[f\left(z\right),z_0\right]=\frac{P\left(z_0\right)}{Q'\left(z_0\right)}
$$

\textbf{拓展:}若极点$z_0$的级数不为$m$,当它的实际级数要比$m$低时,把$m$作为极点$z_0$的级数来计算留数,并不会影响计算结果
\begin{example}
    \begin{eqnarray}
        \mathrm{Res}\left[\frac{z-\sin z}{z^6},0\right]&=&\frac{1}{\left(6-1\right)!}\lim_{z\to 0}{\frac{\mathrm{d^5}}{\mathrm{d}z^5}\left(z^6\cdot\frac{z-\sin z}{z^6}\right)} \nonumber      \\
        ~&=&\frac{1}{5!}\lim_{z\to 0}\left(-\cos \right) \nonumber    \\
		~&=&-\frac{1}{5!} \nonumber
    \end{eqnarray}
\end{example}
\textbf{函数在无穷远点的留数}\\
\hspace*{2em}\textbf{规则4:}
$$
\mathrm{Res}\left[f\left(z\right),\infty\right]=-\mathrm{Res}\left[f\left(\frac{1}{z}\right)\cdot\frac{1}{z^2},0\right]
$$
\newpage
\section{留数定理及其应用}
\begin{theorem}
	设$f\left(z\right)$在以简单闭曲线$C$为边界的闭区域$\overline{D}=D+C$内除$z_1$,$z_2$,$z_3$,$\cdots$,$z_n$外解析,其中$z_1$,$z_2$,$z_3$,$\cdots$,$z_n$为$f\left(z\right)$在$D$内的孤立奇点,则
	$$
	\oint_C{f\left(z\right)\mathrm{d}z}=2\pi\mathrm{i}\cdot\sum_{k=1}^{n}{\mathrm{Res}\left[f\left(z\right),z_k\right]}
	$$
\end{theorem}
\begin{theorem}
	如果函数$f\left(z\right)$在扩充复平面内只有有限个孤立奇点:$z_1$,$z_2$,$z_3$,$\cdots$,$z_n$,$\infty$,则$\left(z\right)$在各奇点的留数总和为零,即
	$$
	\sum_{k=1}^{n}{\mathrm{Res}\left[f\left(z\right),z_k\right]}+\mathrm{Res}\left[f\left(z\right),\infty\right]=0
	$$
\end{theorem}
\textbf{留数定理的应用}\\
(1)计算沿封闭曲线的积分(重点)
\begin{example}
	计算积分
	$$
	\oint_C{\frac{z\mathrm{e}^z}{z^2-1}\mathrm{d}z}
	$$
	\hspace*{2em}其中$C$为正向圆周$\left|z\right|=2$\\
	\hspace*{1em}\textbf{解}\\
	\hspace*{2em}
	$$
	f\left(z\right)=\frac{z\mathrm{e}^z}{z^2-1}
	$$
	\hspace*{2em}在圆$\left|z\right|=2$内有两个一级极点,为$z=\pm \mathrm{1}$且有
	$$
	\mathrm{Res}\left[f\left(z\right),1\right]=\frac{\mathrm{e}}{2},\mathrm{Res}\left[f\left(z\right),-1\right]=\frac{\mathrm{e}^{-1}}{2}
	$$
	\hspace*{2em}因此
	$$
	\oint_C{\frac{z\mathrm{e}^z}{z^2-1}\mathrm{d}z}=2\pi\mathrm{i}\left[\mathrm{Res}\left[f\left(z\right),1\right]+\mathrm{Res}\left[f\left(z\right),-1\right]\right]=2\pi\mathrm{i}\left(\frac{\mathrm{e}}{2}+\frac{\mathrm{e}^{-1}}{2}\right)=\pi\mathrm{i}\left(\mathrm{e}+\frac{1}{\mathrm{e}}\right)
	$$
\end{example}
\newpage

\noindent
(2)在定积分计算中的应用(几乎不考)\\
\hspace*{2em}\textbf{[A]}形如
$$
\int_0^{2\pi}R\left(\cos \theta,\sin \theta\right)\mathrm{d}\theta
$$
\hspace*{3em}的积分,利用$z=\mathrm{e}^{\mathrm{i}\theta}$,$0\leqslant\theta\leqslant 2\pi$,做代换
$$
\left\{ \begin{array}{l}
	\sin \theta =\frac{z^2-1}{2\mathrm{i}z}\\
	\cos \theta =\frac{z^2+1}{2z}\\
\end{array} \right. 
$$
\hspace*{3em}从而化为
$$
\oint_{\left|z\right|=1}{R\left(\frac{z^2+1}{2z},\frac{z^2-1}{2\mathrm{i}z}\right)}\frac{\mathrm{d}z}{\mathrm{i}z}=\oint_{\left|z\right|=1}f\left(z\right)\mathrm{d}z
$$
\hspace*{3em}再依据$\left|z\right|=1$中的孤立奇点,通过留数定理计算\\
\hspace*{2em}\textbf{[B]}形如
$$
\int_{-\infty}^{+\infty}R\left(x\right)\mathrm{d}x=\int_{-\infty}^{+\infty}\frac{P\left(x\right)}{Q\left(x\right)}\mathrm{d}x
$$
\hspace*{3em}的积分,且满足\\
\hspace*{3em}(1) $P\left(x\right)$与$Q\left(x\right)$为互质多项式\\
\hspace*{3em}(2)分母次数至少比分子次数高两次\\
\hspace*{3em}(3) $Q\left(x\right)$在实轴上没有零点\\
\hspace*{3em}我们记$R\left(z\right)=\frac{P\left(z\right)}{Q\left(z\right)}$,设其在上半平面内的所有极点为$z_1$,$z_2$,$\cdots$,$z_k$,则
$$
\int_{-\infty}^{+\infty}R\left(x\right)\mathrm{d}x=2\pi\mathrm{i}\sum_{k}\mathrm{Res}\left[R\left(z\right),z_k\right]
$$
\hspace*{2em}\textbf{[C]}形如
$$
\int_{-\infty}^{+\infty}R\left(x\right)\cdot\mathrm{e}^{\mathrm{i}\alpha x}\mathrm{d}x=\int_{-\infty}^{+\infty}\frac{P\left(x\right)}{Q\left(x\right)}\cdot\mathrm{e}^{\mathrm{i}\alpha x}\mathrm{d}x,\alpha>0
$$
\hspace*{3em}的积分,且满足\\
\hspace*{3em}(1) $R\left(x\right)$为互质多项式\\
\hspace*{3em}(2)分母次数至少比分子次数高一次\\
\hspace*{3em}(3) $R\left(x\right)$在实轴上没有零点\\
\hspace*{3em}我们记$R\left(z\right)=\frac{P\left(z\right)}{Q\left(z\right)}$,设其在上半平面内的所有极点为$z_1$,$z_2$,$\cdots$,$z_k$,则
$$
\int_{-\infty}^{+\infty}R\left(x\right)\cdot\mathrm{e}^{\mathrm{i}\alpha x}\mathrm{d}x=2\pi\mathrm{i}\sum_{k}\mathrm{Res}\left[R\left(z\right)\cdot\mathrm{e}^{\mathrm{i}\alpha z},z_k\right]
$$
\end{document}