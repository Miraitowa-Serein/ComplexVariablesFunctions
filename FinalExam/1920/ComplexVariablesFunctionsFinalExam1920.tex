\documentclass[14pt,a4paper,UTF8]{ctexart}
%\usepackage{ctexcap}
\usepackage{amssymb}
\usepackage{amsmath}
\usepackage[left=1.9cm, right=1.9cm, top=2.5cm, bottom=2.4cm]{geometry}
\usepackage{setspace}
\usepackage{multicol}
\usepackage{fancyhdr}
\usepackage{graphicx}
%\usepackage{color}
\usepackage{diagbox}
%\usepackage{mathpazo}
%\usepackage{pgfplots}
%\pgfplotsset{compat=1.16}
\usepackage{tikz}
\linespread{2.2}
\pagestyle{fancy}
\fancyhf{}
\fancyhead[LE,RO]{2019-2020复变函数}
\fancyhead[RE,LO]{南京工程学院}
\fancyfoot[CE,CO]{\leftmark}
\fancyfoot[LE,RO]{\thepage}

\newcommand{\onech}[4]{
	\hspace*{2em}\makebox[92pt][l]{A.\, #1} \hfill
	\makebox[92pt][l]{B.\, #2} \hfill
	\makebox[92pt][l]{C.\, #3} \hfill
	\makebox[92pt][l]{D.\, #4}\\}
\newcommand{\twoch}[4]{
	\hspace*{2em}\makebox[92pt][l]{A.\, #1} \hfill\makebox[210pt][l]{B.\, #2}\\
	\hspace*{2em}\makebox[92pt][l]{C.\, #3} \hfill\makebox[210pt][l]{D.\, #4}\\}

\renewcommand{\headrulewidth}{1pt}
\renewcommand{\footrulewidth}{0.5pt}
\everymath{\displaystyle}
\begin{document}
	\section*{\textbf{南京工程学院 2019-2020复变函数期末试题}}
	\noindent
	\textbf{一、 单项选择题:} 本题共 4 小题,每小题 6 分,共 24 分。\\
	\textbf{1. } 设复数$z$满足$\mathrm{arg}\left(z+2\right)=\frac{\pi}{3}$,$\mathrm{arg}\left(z-2\right)=\frac{5\pi}{6}$,那么$z=$ \\
	\onech{$-1+\sqrt{3}\mathrm{i}$}{$-\sqrt{3}+\mathrm{i}$}{$-\frac{1}{2}+\frac{\sqrt{3}}{2}\mathrm{i}$}{$-\frac{\sqrt{3}}{2}+\frac{1}{2}\mathrm{i}$}
	\textbf{2. } 函数$f\left(z\right)=z^2\mathrm{Im}\left(z\right)$在$z=0$处的导数为 \\
	\onech{$0$}{$1$}{$-1$}{不存在}
	\textbf{3. } 设$C$是正向圆周$x^2+y^2-2x=0$,则$\oint_C \frac{\sin \frac{\pi}{4}z}{z^2-1}\mathrm{d}z=$ \\
	\onech{$\frac{\sqrt{2}}{2}\pi\mathrm{i}$}{$\sqrt{2}\pi\mathrm{i}$}{$0$}{$-\frac{\sqrt{2}}{2}\pi\mathrm{i}$}
	\textbf{4. } 设函数$\frac{\mathrm{e}^{z}}{\cos z}$的泰勒展开式为$\sum\limits_{n=0}^{+\infty}{c_nz^n}$,那么幂级数$\sum\limits_{n=0}^{+\infty}{c_nz^n}$的收敛半径$R=$ \\
	\onech{$+\infty$}{$1$}{$\frac{\pi}{2}$}{$\pi$}
	\textbf{5. } $\mathrm{i}^\mathrm{i}$的主值为 \\
	\onech{$0$}{$1$}{$\mathrm{e}^{\frac{\pi}{2}}$}{$\mathrm{e}^{-\frac{\pi}{2}}$}
	\textbf{6. } $ z=0 $ 是函数 $ \frac{1-\mathrm{e}^z}{z^4\sin z} $的\\
	\onech{5级极点}{4级极点}{可去奇点}{3级极点}
	\\
	\textbf{二、 填空题:} 本题共 6 小题,每小题 4 分,共 24 分。\\
	\textbf{1. } $ \left(-1+\mathrm{i}\right)^8 = $ \underline{\hspace*{6em}}。\\
	\textbf{2. } $ \lim\limits_{z\to1}{\frac{z\overline{z}+3z-\overline{z}-3}{z^2-1}} = $ \underline{\hspace*{6em}}。\\ 
	\textbf{3. } 设$f\left(z\right)=x^3+y^3+\mathrm{i}x^2y^2$,则$f'\left(-\frac{3}{2}+\frac{3}{2}\mathrm{i}\right) = $ \underline{\hspace*{6em}}。\\
	\textbf{4. } 设$C$为正向圆周$\left|z-4\right|=1$,则$ \oint_C{\frac{z^2-3z+2}{\left(z-4\right)^2}\mathrm{d}z}= $ \underline{\hspace*{6em}}。\\
	\textbf{5. } 设$\frac{1}{z^2+1}=\sum\limits_{n=-\infty}^{+\infty}c_nz^n$,$\left|z\right|>1$,则$c_{-6} = $ \underline{\hspace*{6em}}。\\
	\textbf{6. } 设$ f\left(z\right)=\frac{\mathrm{e}^z}{z^2\left(1-z\right)} $,则$\mathrm{Res}\left[f\left(z\right),0\right] = $ \underline{\hspace*{6em}}。\\
    \\
    \textbf{三、计算题:} 本题共 5 小题,每小题 8 分,共 40 分。\\
	\textbf{1. } 若复数$z$满足方程
    $$
	z+\left|\overline{z}\right|=2+\mathrm{i}
	$$
	求复数$z$
	\\[20pt]
	\textbf{2. } 设函数
    $$
    f\left(z\right)=z^2\cdot\overline{z}
    $$
    试判断$f\left(z\right)$在何处可导,何处解析,并计算可导点处的导数
	\\[20pt]
    \textbf{3. } 将函数
    $$
    f\left(z\right)=\frac{z}{\left(z+1\right)\left(z+2\right)}
    $$
    在$z=1$处展开成泰勒级数,并指出其收敛区域
	\\[20pt]
	\textbf{4. } 设
    $$
    f\left(z\right)=\frac{1}{z^2\left(z-\mathrm{i}\right)^3}
    $$
    试在$\left| z-\mathrm{i} \right|>1$内将$f\left(z\right)$展开成洛朗级数
	\\[20pt]
	\textbf{5. } 设$C$为正向圆周$\left|z\right|=4$,计算积分
    $$
    \oint_C{\frac{\mathrm{e}^z}{\left(z-1\right)\left(z-2\right)^2}\mathrm{d}z}
    $$
	\\
    \textbf{四、应用题:} 本题共 1 小题,每小题 12 分,共 12 分。\\
    \textbf{1. } 已知函数
	$$
	v\left(x,y\right)=\arctan\frac{y}{x},x>0
	$$
	解析,求解析函数$f\left(z\right)=u+\mathrm{i}v$
 
\end{document}           